\documentclass[../../main.tex]{subfiles}

\begin{document}

\subsection{7.1 Keys and Foreign Keys}

\subsubsection*{Exercise 1.1}

a)

\begin{lstlisting}[language=sql]
  FOREIGN KEY (producerC#) REFERENCES MovieExec(cert#);
\end{lstlisting}

b)

\begin{lstlisting}[language=sql]
  FOREIGN KEY (producerC#) REFERENCES MovieExec(cert#)
    ON UPDATE SET NULL;
\end{lstlisting}

c)

\begin{lstlisting}[language=sql]
  FOREIGN KEY (producerC#) REFERENCES MovieExec(cert#)
    ON UPDATE SET CASCADE
    ON DELETE SET CASCADE;
\end{lstlisting}

d)

\begin{lstlisting}[language=sql]
  FOREIGN KEY (movieTitle, movieYear)
    REFERENCES Movies(title, year);
\end{lstlisting}

e)

\begin{lstlisting}[language=sql]
  FOREIGN KEY (starName) REFERENCES MovieStar(name)
    ON DELETE CASCADE;
\end{lstlisting}

\subsubsection*{Exercise 1.2}

No, A foreign key must refer to the primary key in some
relation. But \verb|movieTitle| and \verb|movieYear|
are not a key for \verb|StarsIn|.

\end{document}
